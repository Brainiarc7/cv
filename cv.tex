%% start of file `cv.tex'.
%% Based on `template.tex` from the moderncv distribution by Xavier Danaux
%
% This work may be distributed and/or modified under the
% conditions of the LaTeX Project Public License version 1.3c,
% available at http://www.latex-project.org/lppl/.


\documentclass[11pt,a4paper,sans]{moderncv}   % possible options include font size ('10pt', '11pt' and '12pt'), paper size ('a4paper', 'letterpaper', 'a5paper', 'legalpaper', 'executivepaper' and 'landscape') and font family ('sans' and 'roman')

% moderncv themes
\moderncvstyle{classic}                        % style options are 'casual' (default), 'classic', 'oldstyle' and 'banking'
\moderncvcolor{blue}                          % color options 'blue' (default), 'orange', 'green', 'red', 'purple', 'grey' and 'black'
%\renewcommand{\familydefault}{\sfdefault}    % to set the default font; use '\sfdefault' for the default sans serif font, '\rmdefault' for the default roman one, or any tex font name
\nopagenumbers{}                             % uncomment to suppress automatic page numbering for CVs longer than one page

% character encoding
%\usepackage[utf8]{inputenc}                  % if you are not using xelatex ou lualatex, replace by the encoding you are using
%\usepackage{CJKutf8}                         % if you need to use CJK to typeset your resume in Chinese, Japanese or Korean

% adjust the page margins
\usepackage[scale=0.75]{geometry}
%\setlength{\hintscolumnwidth}{3cm}           % if you want to change the width of the column with the dates
%\setlength{\makecvtitlenamewidth}{10cm}      % for the 'classic' style, if you want to force the width allocated to your name and avoid line breaks. be careful though, the length is normally calculated to avoid any overlap with your personal info; use this at your own typographical risks...

% personal data
\firstname{Alan}
\familyname{Orth}
%\title{A great man}                          % optional, remove / comment the line if not wanted
% \address{Nairobi}{Kenya}    % optional, remove / comment the line if not wanted
\mobile{+254 726 388 307}                     % optional, remove / comment the line if not wanted
\email{aorth@thefro.org}                          % optional, remove / comment the line if not wanted
\homepage{github.com/alanorth}                    % optional, remove / comment the line if not wanted
%\extrainfo{additional information}            % optional, remove / comment the line if not wanted
%\quote{The quieter you become, the more you are able to hear...}                            % optional, remove / comment the line if not wanted

% to show numerical labels in the bibliography (default is to show no labels); only useful if you make citations in your resume
%\makeatletter
%\renewcommand*{\bibliographyitemlabel}{\@biblabel{\arabic{enumiv}}}
%\makeatother

% bibliography with mutiple entries
%\usepackage{multibib}
%\newcites{book,misc}{{Books},{Others}}
%----------------------------------------------------------------------------------
%            content
%----------------------------------------------------------------------------------
\begin{document}
%\begin{CJK*}{UTF8}{gbsn}                     % to typeset your resume in Chinese using CJK
%-----       resume       ---------------------------------------------------------
\makecvtitle

\section{Education}
\cventry{2002--2007}{Computer Information Systems}{California State University}{Chico}{\textit{3.19}}{Emphasis on Computer Science, minor in Business}  % arguments 3 to 6 can be left empty

\section{Experience}
\subsection{Vocational}
\cventry{2011--current}{Systems Analyst}{International Livestock Research Institute}{Nairobi, Kenya}{}{Oversee research computing infrastructure and support scientists\newline{}
  \begin{itemize}
  \item Design and implement next-generation storage and computing platform using GlusterFS, SLURM and 389 LDAP
  \item Deploy KVM virtual machines for students, web applications, and test environments
  \item Implement and maintain tape backups of research data using Amanda Backup
  \item Maintain and deploy source code customizations of the \href{http://cgspace.cgiar.org}{CGIAR DSpace} institutional repository
  \item Install and configure bioinformatics applications like BLAST+, Galaxy, structure, and R
  \item Deploy field data collection system using Android and ODK Collect/Aggregate
  \end{itemize}}
\cventry{2009--2011}{Linux Consultant}{International Livestock Research Institute}{Nairobi, Kenya}{}{{}
  \begin{itemize}
  \item Administer Rocks Linux cluster and bioinformatics applications
  \item Provide technical support on network and server issues to Biotechnology group
  \item Administer physical and virtual servers for students and researchers
  \item Develop modules to supplement functionality of off-the-shelf LIMS
  \item Develop standard operating procedures and train staff in their use
  \end{itemize}}
\cventry{2007--2009}{Lecturer and Systems Administrator}{Holy Rosary College}{Tala, Kangundo}{}{Volunteer with VSO Jitolee, Kenya\newline{}
\begin{itemize}
\item Manage and update Linux servers running Squid proxy, Apache, and Samba file services
\item Troubleshoot satellite modem, routers, network cables, and LAN switches
\item Maintain workstations for staff and student computer lab
\item Taught courses:
  \begin{itemize}
  \item Intro to Programming and Algorithms (C++)
  \item Network Essentials
  \item Object-oriented Programming (C++)
  \item Operating Systems II (Linux)
  \end{itemize}
\end{itemize}}
\cventry{2004--2007}{Systems Administrator}{Upward Bound Organization}{Chico, CA}{}{{}
\begin{itemize}
\item Manage Mac OS X servers running Apache, Mailman, and OpenLDAP
\item Manage student computer lab and network-attached student home directories
\item Monitor security of public-facing servers, including remote administration over SSH
\item Technical support and maintenance of staff workstations
\item Write documentation of configuration and maintenance procedures for servers
\item Interview and train replacement systems administrator upon my departure
\end{itemize}}

\subsection{Internships}
\cventry{June, 2006}{Systems Administrator}{Chevron Corporation}{San Ramon, CA}{}{Three-month internship with the Technical Computing team\newline{}
\begin{itemize}
\item Write interactive Perl script to help administrators maintain synchronized application repositories on distributed Linux and Solaris servers
\item Write interactive Perl script to ease installation and configuration of high-performance cluster nodes using SystemImager
\end{itemize}}

\cventry{June, 2005}{Systems Analyst}{Chevron Corporation}{San Ramon, CA}{}{Three-month internship with the Application Server Design team\newline{}
\begin{itemize}
\item Install and support Windows 2000/2003 servers and their web applications (.NET)
\item Write documentation in preparation for annual team audit (SOX)
\item Document F5 BIG-IP DNS settings used for fail over and geographic load balancing for internal and external corporate web applications
\end{itemize}}

\section{Computer skills}
\cvitem{Proficient}{vim, KVM, git, bash, regular expressions, CSS, Apache httpd, mod\_rewrite}
\cvitem{Intermediate}{MySQL, PostgreSQL, PHP, Apache Tomcat, gerrit}
\cvitem{Padawan}{\LaTeX}

\section{Languages}
\cvitemwithcomment{English}{Native}{Written and spoken}
\cvitemwithcomment{Swahili}{Intermediate}{Written and spoken}

\section{Interests}
\cvitem{Android}{Porting Android to new devices}
\cvitem{Open-source}{Contributing to open-source software projects}
\cvitem{Community}{\href{http://nairobilug.or.ke}{Nairobi Linux Users Group}}
\cvitem{Blogging}{\href{http://alaninkenya.org}{Personal adventures in Kenya}, \href{http://mjanja.co.ke}{Adventures in "underground" tech}}
\cvitem{GNU/Linux}{Arch Linux, Debian, Gentoo, CentOS, Ubuntu}

%\renewcommand{\listitemsymbol}{-~}            % change the symbol for lists

% Publications from a BibTeX file without multibib
%  for numerical labels: \renewcommand{\bibliographyitemlabel}{\@biblabel{\arabic{enumiv}}}
%  to redefine the heading string ("Publications"): \renewcommand{\refname}{Articles}
%\nocite{*}
%\bibliographystyle{plain}
%\bibliography{publications}                   % 'publications' is the name of a BibTeX file

% Publications from a BibTeX file using the multibib package
%\section{Publications}
%\nocitebook{book1,book2}
%\bibliographystylebook{plain}
%\bibliographybook{publications}              % 'publications' is the name of a BibTeX file
%\nocitemisc{misc1,misc2,misc3}
%\bibliographystylemisc{plain}
%\bibliographymisc{publications}              % 'publications' is the name of a BibTeX file
\end{document}


%% end of file `cv.tex'.

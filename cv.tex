%% start of file `cv.tex'.
%% Based on `template.tex` from the moderncv distribution by Xavier Danaux
%
% This work may be distributed and/or modified under the
% conditions of the LaTeX Project Public License version 1.3c,
% available at http://www.latex-project.org/lppl/.


\documentclass[11pt,a4paper,sans]{moderncv}   % possible options include font size ('10pt', '11pt' and '12pt'), paper size ('a4paper', 'letterpaper', 'a5paper', 'legalpaper', 'executivepaper' and 'landscape') and font family ('sans' and 'roman')

% moderncv themes
\moderncvstyle{classic}                        % style options are 'casual' (default), 'classic', 'oldstyle' and 'banking'
\moderncvcolor{blue}                          % color options 'blue' (default), 'orange', 'green', 'red', 'purple', 'grey' and 'black'
%\renewcommand{\familydefault}{\sfdefault}    % to set the default font; use '\sfdefault' for the default sans serif font, '\rmdefault' for the default roman one, or any tex font name
%\nopagenumbers{}                             % uncomment to suppress automatic page numbering for CVs longer than one page

% character encoding
%\usepackage[utf8]{inputenc}                  % if you are not using xelatex ou lualatex, replace by the encoding you are using
%\usepackage{CJKutf8}                         % if you need to use CJK to typeset your resume in Chinese, Japanese or Korean

% adjust the page margins
\usepackage[scale=0.75]{geometry}
%\setlength{\hintscolumnwidth}{3cm}           % if you want to change the width of the column with the dates
%\setlength{\makecvtitlenamewidth}{10cm}      % for the 'classic' style, if you want to force the width allocated to your name and avoid line breaks. be careful though, the length is normally calculated to avoid any overlap with your personal info; use this at your own typographical risks...

% personal data
\firstname{Alan}
\familyname{Orth}
\title{A great man}                          % optional, remove / comment the line if not wanted
% \address{Nairobi}{Kenya}    % optional, remove / comment the line if not wanted
\mobile{+254 726 388 307}                     % optional, remove / comment the line if not wanted
\email{aorth@thefro.org}                          % optional, remove / comment the line if not wanted
\homepage{mjanja.co.ke}                    % optional, remove / comment the line if not wanted
\extrainfo{additional information}            % optional, remove / comment the line if not wanted
\quote{ROOT ALL THE BOXES!}                            % optional, remove / comment the line if not wanted

% to show numerical labels in the bibliography (default is to show no labels); only useful if you make citations in your resume
%\makeatletter
%\renewcommand*{\bibliographyitemlabel}{\@biblabel{\arabic{enumiv}}}
%\makeatother

% bibliography with mutiple entries
%\usepackage{multibib}
%\newcites{book,misc}{{Books},{Others}}
%----------------------------------------------------------------------------------
%            content
%----------------------------------------------------------------------------------
\begin{document}
%\begin{CJK*}{UTF8}{gbsn}                     % to typeset your resume in Chinese using CJK
%-----       resume       ---------------------------------------------------------
\makecvtitle

\section{Education}
\cventry{2002--2007}{Computer Information Systems}{California State University}{Chico}{\textit{3.19}}{Emphasis on Computer Science, minor in Business}  % arguments 3 to 6 can be left empty

\section{Experience}
\subsection{Vocational}
\cventry{2011--current}{Systems Analyst}{International Livestock Research Institute}{Nairobi, Kenya}{}{Oversee research computing infrastructure and support scientists\newline{}%
  \begin{itemize}%
  \item Achievement 1;
  \item Achievement 2, with sub-achievements:
    \begin{itemize}%
    \item Sub-achievement (a);
    \end{itemize}
  \item Achievement 3.
  \end{itemize}}
\cventry{2009--2011}{Linux Consultant}{International Livestock Research Institute}{Nairobi, Kenya}{}{Description line 1\newline{}Description line 2}
\cventry{2007--2009}{Lecturer and Systems Administrator}{Holy Rosary College}{Tala, Kangundo}{}{Volunteer with VSO Jitolee, Kenya\newline{}%
\begin{itemize}%
\item Manage and update Linux servers running Squid proxy, Apache, and Samba file services
\item Troubleshoot satellite modem, routers, network cables, and LAN switches
\item Maintain workstations for staff and student computer lab
\item Taught courses:%
  \begin{itemize}%
  \item Intro to Programming and Algorithms (C++)
  \item Network Essentials
  \item Object-oriented Programming (C++)
  \item Operating Systems II (Linux)
  \end{itemize}
\end{itemize}}
\subsection{Miscellaneous}
\cventry{year--year}{Job title}{Employer}{City}{}{Description}

\section{Languages}
\cvitemwithcomment{Language 1}{Skill level}{Comment}
\cvitemwithcomment{Language 2}{Skill level}{Comment}
\cvitemwithcomment{Language 3}{Skill level}{Comment}

\section{Computer skills}
\cvdoubleitem{category 1}{XXX, YYY, ZZZ}{category 4}{XXX, YYY, ZZZ}
\cvdoubleitem{category 2}{XXX, YYY, ZZZ}{category 5}{XXX, YYY, ZZZ}
\cvdoubleitem{category 3}{XXX, YYY, ZZZ}{category 6}{XXX, YYY, ZZZ}

\section{Interests}
\cvitem{Android}{Porting Android to new devices}
\cvitem{Open-source software}{Contributing to open-source software projects}
\cvitem{hobby 3}{Description}

\section{Extra 1}
\cvlistitem{Item 1}
\cvlistitem{Item 2}
\cvlistitem{Item 3}

\renewcommand{\listitemsymbol}{-~}            % change the symbol for lists

\section{Extra 2}
\cvlistdoubleitem{Item 1}{Item 4}
\cvlistdoubleitem{Item 2}{Item 5\cite{book1}}
\cvlistdoubleitem{Item 3}{}

% Publications from a BibTeX file without multibib
%  for numerical labels: \renewcommand{\bibliographyitemlabel}{\@biblabel{\arabic{enumiv}}}
%  to redefine the heading string ("Publications"): \renewcommand{\refname}{Articles}
\nocite{*}
\bibliographystyle{plain}
\bibliography{publications}                   % 'publications' is the name of a BibTeX file

% Publications from a BibTeX file using the multibib package
%\section{Publications}
%\nocitebook{book1,book2}
%\bibliographystylebook{plain}
%\bibliographybook{publications}              % 'publications' is the name of a BibTeX file
%\nocitemisc{misc1,misc2,misc3}
%\bibliographystylemisc{plain}
%\bibliographymisc{publications}              % 'publications' is the name of a BibTeX file
\end{document}


%% end of file `cv.tex'.
